% Options for packages loaded elsewhere
\PassOptionsToPackage{unicode}{hyperref}
\PassOptionsToPackage{hyphens}{url}
%
\documentclass[
  ignorenonframetext,
]{beamer}
\usepackage{pgfpages}
\setbeamertemplate{caption}[numbered]
\setbeamertemplate{caption label separator}{: }
\setbeamercolor{caption name}{fg=normal text.fg}
\beamertemplatenavigationsymbolsempty
% Prevent slide breaks in the middle of a paragraph
\widowpenalties 1 10000
\raggedbottom
\setbeamertemplate{part page}{
  \centering
  \begin{beamercolorbox}[sep=16pt,center]{part title}
    \usebeamerfont{part title}\insertpart\par
  \end{beamercolorbox}
}
\setbeamertemplate{section page}{
  \centering
  \begin{beamercolorbox}[sep=12pt,center]{part title}
    \usebeamerfont{section title}\insertsection\par
  \end{beamercolorbox}
}
\setbeamertemplate{subsection page}{
  \centering
  \begin{beamercolorbox}[sep=8pt,center]{part title}
    \usebeamerfont{subsection title}\insertsubsection\par
  \end{beamercolorbox}
}
\AtBeginPart{
  \frame{\partpage}
}
\AtBeginSection{
  \ifbibliography
  \else
    \frame{\sectionpage}
  \fi
}
\AtBeginSubsection{
  \frame{\subsectionpage}
}
\usepackage{lmodern}
\usepackage{longtable}
\usepackage{amssymb,amsmath}
\usepackage{ifxetex,ifluatex}
\ifnum 0\ifxetex 1\fi\ifluatex 1\fi=0 % if pdftex
  \usepackage[T1]{fontenc}
  \usepackage[utf8]{inputenc}
  \usepackage{textcomp} % provide euro and other symbols
\else % if luatex or xetex
  \usepackage{unicode-math}
  \defaultfontfeatures{Scale=MatchLowercase}
  \defaultfontfeatures[\rmfamily]{Ligatures=TeX,Scale=1}
\fi
% Use upquote if available, for straight quotes in verbatim environments
\IfFileExists{upquote.sty}{\usepackage{upquote}}{}
\IfFileExists{microtype.sty}{% use microtype if available
  \usepackage[]{microtype}
  \UseMicrotypeSet[protrusion]{basicmath} % disable protrusion for tt fonts
}{}
\makeatletter
\@ifundefined{KOMAClassName}{% if non-KOMA class
  \IfFileExists{parskip.sty}{%
    \usepackage{parskip}
  }{% else
    \setlength{\parindent}{0pt}
    \setlength{\parskip}{6pt plus 2pt minus 1pt}}
}{% if KOMA class
  \KOMAoptions{parskip=half}}
\makeatother
\usepackage{xcolor}
\IfFileExists{xurl.sty}{\usepackage{xurl}}{} % add URL line breaks if available
\IfFileExists{bookmark.sty}{\usepackage{bookmark}}{\usepackage{hyperref}}
\hypersetup{
  pdftitle={Final Presentation},
  pdfauthor={Luke Denton},
  hidelinks,
  pdfcreator={LaTeX via pandoc}}
\urlstyle{same} % disable monospaced font for URLs
\newif\ifbibliography
\setlength{\emergencystretch}{3em} % prevent overfull lines
\providecommand{\tightlist}{%
  \setlength{\itemsep}{0pt}\setlength{\parskip}{0pt}}
\setcounter{secnumdepth}{-\maxdimen} % remove section numbering

\title{An Examination of Secondary School Success Indicators}
\author{Luke Denton}
\date{5/4/2021}

\begin{document}
\frame{\titlepage}

\begin{frame}{Introduction}
\protect\hypertarget{introduction}{}

\begin{itemize}
\tightlist
\item
  Public education is heralded as an opportunity to overcome
  generational poverty
\item
  Difficult to quantify causes of student success
\item
  Insufficient resources
\item
  Best method for student success is unclear
\item
  Inferior educational outcomes for students
\end{itemize}

\end{frame}

\begin{frame}{Data}
\protect\hypertarget{data}{}

UCI Machine Learning - Student Performance Database

\begin{itemize}
\tightlist
\item
  649 observations, students ages 15-22
\item
  1,2,3 period grades in math \& portuguese
\item
  student demographics \& school activities
\item
  parent education, family size, \# failed classes, \# absences, urban
  or rural housing, study time, extracurricular activities, internet
  access, romantic relationships, pursuing higher ed, alcohol
  consumption
\end{itemize}

\end{frame}

\begin{frame}{Data}
    \includegraphics[width = \linewidth]{Rplot02.png}
\end{frame}

\begin{frame}{Methods}
\protect\hypertarget{methods}{}

Multiple Linear Regression to explain causality, in progress using ML to
predict outcomes

\ final = \beta_0 + \beta_1G1 + \beta_2G2 + u \

\ final = \beta_0 + \beta_1G1 + \beta_2G2 +\beta_3Medu + \beta_4Fedu + \beta_5studytime + u \

\ final = \beta_0 + \beta_1G1 + \beta_2G2 + \beta_3Medu + \beta_4Fedu + \beta_5studytime +
        \beta_6address + \beta_7famsize + \beta_8Pstatus + \beta_9activities + u \

\end{frame}

\begin{frame}{Findings}
   \begin{table}
\centering
\begin{tabular}[t]{lc}
\toprule
  & Model 1\\
\midrule
(Intercept) & -1.830\\
 & (0.335)\\
G1 & 0.153\\
 & (0.056)\\
G2 & 0.987\\
 & (0.050)\\
\midrule
Num.Obs. & 395\\
R2 & 0.822\\
R2 Adj. & 0.821\\
AIC & 1648.2\\
BIC & 1664.1\\
Log.Lik. & -820.115\\
F & 906.134\\
\bottomrule
\end{tabular}
\end{table}

\end{frame}

\begin{frame}{Findings}
\protect\hypertarget{findings}{}

\begin{itemize}
\tightlist
\item
  1st and 2nd period grades have a significant causal effect on final
  grades (Rsq = 0.822)

  \begin{itemize}
  \tightlist
  \item
    2nd period grade had a much stronger effect (beta = 0.987)
  \end{itemize}
\item
  No other variables in other models were statistically significant
\end{itemize}

\end{frame}

\begin{frame}{Conclusion}
\protect\hypertarget{conclusion}{}

\begin{itemize}
\tightlist
\item
  Many of the factors attributed to student success did not have a causal effect in this model
\item
  ML model would prove useful for prediction, since statistically significant causality is difficult to establish
\item
  Continued study: using US Dept of Ed database on student performance
\end{itemize}

\end{frame}

\end{document}
