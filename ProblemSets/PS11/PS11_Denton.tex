\documentclass[12pt,english]{article}
\usepackage[utf8]{inputenc}
\usepackage[english]{babel}
\usepackage{color}
\usepackage[dvipsnames]{xcolor}
\definecolor{darkblue}{RGB}{0.,0.,139.}

\usepackage[top=1in, bottom=1in, left=1in, right=1in]{geometry}

\usepackage{amsmath}
\usepackage{amstext}
\usepackage{amssymb}
\usepackage{setspace}
\usepackage{lipsum}

\usepackage[authoryear]{natbib}
\usepackage{url}
\usepackage{booktabs}
\usepackage[flushleft]{threeparttable}
\usepackage{graphicx}
\usepackage[english]{babel}
\usepackage{pdflscape}
\usepackage[unicode=true,pdfusetitle,
 bookmarks=true,bookmarksnumbered=false,bookmarksopen=false,
 breaklinks=true,pdfborder={0 0 0},backref=false,
 colorlinks,citecolor=black,filecolor=black,
 linkcolor=black,urlcolor=black]
 {hyperref}
\usepackage[all]{hypcap} % Links point to top of image, builds on hyperref
\usepackage{breakurl}    % Allows urls to wrap, including hyperref

\linespread{2}

\begin{document}

\begin{singlespace}
\title{An Examination of Secondary School Success Indicators}
\end{singlespace}

\author{Luke Denton\thanks{Department of Economics, University of Oklahoma.\
email~address:~\href{mailto:lukedenton@ou.edu}{lukedenton@ou.edu}}}

% \date{\today}
\date{April 27, 2021}

\maketitle

\begin{abstract}
\begin{singlespace}
In this project, I seek to answer a difficult question many experts and organizations have been asking for years: What drives academic performance for students in secondary school? To answer this question, I have computed linear regressions to better understand the impact of different variables on student grade performance. With currently washy results, the implications of this analysis are still in need of additional research to find causal relationships between student demographics and performance.
\end{singlespace}

\end{abstract}
\vfill{}


\pagebreak{}


\section{Introduction}\label{sec:intro}
Education is one of the biggest components of adolescent life. School is where many children spend thousands of hours learning and interacting with teachers, administrators, and their peers. Heralded as a wonderful opportunity to surpass generational poverty, public education provides schooling for millions of students around the world. Despite the universal existence of public education, a single strategy or set of strategies that help students succeed better in school have largely gone undefined. This drains administrative resources and forces school districts as well as parents to continue to seek the best strategy to foster success for school age children. Without a clear strategy for how parents or teachers can best support children in school, children in schools that are poorly funded are susceptible to a lower quality education. 

\pagebreak{}

\section{Literature Review}\label{sec:litreview}
Much work has been done to better understand the factors that go into student achievement at school. Scholars have studied numerous inputs and scenarios to understand what makes a student succeed in their grades. \citet{emotional} Suggests that emotional intelligence, a fairly difficult trait to quantify, serves as a useful indicator of how well students do in secondary school. \citet{capital} Suggests that higher amounts of capital in both schools and at home produce positive effects for student academic achievement, suggesting that income may not be as large of a definitive indicator of kids' grades. \citet{preparation} Takes an alternative approach and attempts to assess student performance from the teacher side of the classroom. Teacher preparation is something that varies from school to school and may be an overlooked indicator for how well students do in school. Similarly focusing on teachers, \citet{race} focuses on the effects of racial pairings of students and teachers on school performance.  
\pagebreak{}

\section{Data}\label{sec:data}
The primary data source for this research is the student performance database from UCI Machine Learning Data Sets \citet{cortez_silva_2008}. The data tracks secondary school students at two different schools in Portugal. The data measure the first, second, and final period grades of students in mathematics and Portuguese classes, as well as many other demographics concerning home life and school participation. There are 649 total observations in this data set. For an explained list of all the variable codes, see \ref{table:2}.
\pagebreak{}

\section{Empirical Methods}\label{sec:methods}
While I humbly await my office hours with you to help guide me through getting some of the ML stuff sorted out, the main tactic I have engaged to better understand these indicators is a multiple linear regression model. The basic equation for the multiple lm is shown as follows: 
\begin{equation}
        Y = \beta{X} + u
\end{equation} In this analysis, the dependent variable (Y) that I am choosing to focus on (and hopefully be able to predict) is G3, a student's final grade recorded in their class. The independent variables (X's) vary depending on the model, but are included to help provide an explanation for the variation in final grades (Y) among students.

 \begin{itemize}
        \item Model 1: Multiple Linear Regression Model
        
This initial model attempts to explain variation in final grades based off of numerical variables that are typically seen as indicative of a student's performance. 
\begin{equation}
    final = \beta_0 + \beta_1Medu + \beta_2Fedu + \beta_3studytime + u
\end{equation}
\item Model 2: Expanded Multiple Linear Regression Model

This model contains the previous independent variables but also includes new ones in hopes of increasing the amount of variation explained. 
\begin{equation}
    final = \beta_0 + \beta_1Medu + \beta_2Fedu + \beta_3studytime + \beta_4address \beta_5famsize + \beta_6Pstatus + \beta_7activities + u
\end{equation}

\item Model 3: 

This is where I want to dive into ML and attempt to craft a model that penalizes for complexity but does well at predicting out of sample. Need some help from you to make this happen though.
\end{itemize}
\pagebreak{}

\section{Research Findings}\label{sec:results}
The results for models 1 and 2 can be reviewed in \ref{table:1}. From the first model, we learn the substantial importance of mother education level. It's important to keep in mind that the final grades in this data set are on a scale of 20 points; so, an increase of 0.794 per increase in education level of a student's mother is a very significant level of change. Father's education level and also amount of time spent studying each week also have positive effects on students' final grades. These are all rather intuitive relationships that help portray some aspects of students' educational ability that are difficult to quantify (ex: well educated parents are likely to be well-abled and have well-abled children). 

From the second model, there are a few interesting relationships within the data. First, there is a significant positive effect in families with size less than or equal to 3. This indicates that children with no siblings perform better in school. Additionally, the relationship between having parents living together and school performance is negative. There is also a negative relationship between being involved in extracurricular activities and final grade performance. 

These results are not very strong. The R squared of model 1 was a mere 0.055, and model 2 was only 0.070. There is a significant portion of the variation in final grades that remains unexplained from these models. The variance within each variable's estimated effect is also very high. 
\pagebreak{}

\section{Conclusion}\label{sec:conclusion}
The arduous and seemingly mystical task of understanding what precisely causes a student's performance in school is a subject of careful study and mixed reviews. In an environment where factors from home, school, teachers, and peers all interact, predicting education performance as well as finding causal relationships between indicators is difficult. Beyond intuitive indicators that weren't even statistically significant, there is much more that needs to be done in the scope of this project. 

\vfill
\pagebreak{}
\begin{spacing}{1.0}
\bibliographystyle{jpe}
\nocite{*}
\bibliography{PS11_Denton.bib}
\addcontentsline{toc}{section}{References}
\end{spacing}

\vfill
\pagebreak{}
\clearpage

%========================================
% FIGURES AND TABLES 
%========================================
\section{Figures and Tables}
%----------------------------------------
% Table 1
%----------------------------------------
\begin{table}[h]
\begin{center}
\centering
\begin{tabular}[t]{lcc}
\toprule
  & Model 1 & Model 2\\
\midrule
(Intercept) & 6.907 & 6.249\\
 & (0.833) & (1.205)\\
Medu & 0.794 & 0.742\\
 & (0.264) & (0.267)\\
Fedu & 0.147 & 0.180\\
 & (0.265) & (0.265)\\
studytime & 0.468 & 0.526\\
 & (0.269) & (0.271)\\
addressU &  & 0.794\\
 &  & (0.547)\\
famsizeLE3 &  & 0.918\\
 &  & (0.505)\\
PstatusT &  & -0.253\\
 &  & (0.756)\\
activitiesyes &  & -0.103\\
 &  & (0.458)\\
 Walc &  & -0.082\\
 &  & (0.183)\\
\midrule
Num.Obs. & 395 & 395\\
R2 & 0.055 & 0.070\\
R2 Adj. & 0.048 & 0.053\\
AIC & 2310.0 & 2311.6\\
BIC & 2329.9 & 2347.4\\
Log.Lik. & -1150.017 & -1146.787\\
F & 7.576 & 4.178\\
\bottomrule
\end{tabular}
\caption{Model Summary Output}
\label{table:1}
\end{center}
\end{table}
\pagebreak{}

%----------------------------------------
% Table 2
%----------------------------------------
Table 2 is an explanation of the variables used in this analysis.
\begin{table}[h!]
\begin{center}
 \begin{tabular}{|c|c|} 
 \hline
 variable & description  \\ [0.5ex] 
 \hline\hline
 school & student's school \\ 
 \hline 
 sex & student's sex \\
 \hline 
age & student's age \\
\hline  
address & student's home address type\\
 \hline 
 famsize & family size \\
 \hline 
 Pstatus & parent's cohabitation status \\
\hline  
Medu & mother's education \\
\hline  
Fedu & father's education\\
 \hline 
 Mjob & mother's job \\
 \hline 
 Fjob & father's job \\
\hline 
reason & reason to choose this school\\
\hline 
guardian & student's guardian\\
\hline 
traveltime & home to school travel time\\
\hline 
studytime & weekly study time\\
\hline 
failures & number of past class failures\\
\hline 
schoolsup & extra educational support \\
\hline 
famsup & family educational support \\
\hline 
paid & extra paid classes within the course subject  \\
\hline 
activities & extra-curricular activities \\
\hline 
nursery & attended nursery school\\
\hline 
higher & wants to take higher education \\
\hline 
 internet & Internet access at home \\
 \hline 
romantic & with a romantic relationship \\
\hline 
famrel & quality of family relationships \\
\hline 
freetime & free time after school\\ 
\hline 
goout & going out with friends \\
\hline 
Dalc & workday alcohol consumption  \\
\hline
Walc & weekend alcohol consumption\\
\hline 
health & current health status \\
\hline 
absences & number of school absences\\
\hline 
final & final grade\\
\hline
\end{tabular}
\caption{variable codes and explanations}
\label{table:2}
\end{center}
\end{table}

\end{document}
